\documentclass[11pt]{article}
\usepackage{amsmath}
\usepackage{amsfonts} 
\usepackage[english]{babel}
\usepackage{amsthm}
\usepackage{mathtools}
\usepackage{subcaption}
\usepackage{hyperref}
\usepackage{algorithmic}
\usepackage{algorithm}
\usepackage{anyfontsize} % fix font size warning. 
\usepackage{url} 
\urlstyle{same} % fix wacky url links in bib entries. 
% \usepackage{minted}
\usepackage[switch, displaymath, mathlines]{lineno}



% Basic Type Settings ----------------------------------------------------------
\usepackage[margin=1in,footskip=0.25in]{geometry}
\linespread{1}  % double spaced or single spaced
\usepackage[fontsize=12pt]{fontsize}
\usepackage{authblk}

\theoremstyle{definition}
\newtheorem{theorem}{Theorem}       % Theorem counter global 
\newtheorem{prop}{Proposition}[section]  % proposition counter is section
\newtheorem{claim}{Claim}[subsection]
\newtheorem{lemma}{Lemma}[subsection]  % lemma counter is subsection
\newtheorem{definition}{Definition}
\newtheorem{remark}{Remark}[subsection]
{
    % \theoremstyle{plain}
    \newtheorem{assumption}{Assumption}
}
\numberwithin{equation}{subsection}

\hypersetup{
    colorlinks=true,
    linkcolor=blue,
    filecolor=magenta,
    urlcolor=cyan,
}
\usepackage[final]{graphicx}
\usepackage{listings}
\usepackage{courier}
\lstset{basicstyle=\footnotesize\ttfamily,breaklines=true}
\newcommand{\indep}{\perp \!\!\! \perp}
\usepackage{wrapfig}
\graphicspath{{.}}
\usepackage{fancyvrb}

%%
%% Julia definition (c) 2014 Jubobs
%%
\usepackage[T1]{fontenc}
\usepackage{beramono}
\usepackage[usenames,dvipsnames]{xcolor}
\lstdefinelanguage{Julia}%
  {morekeywords={abstract,break,case,catch,const,continue,do, else, elseif,%
      end, export, false, for, function, immutable, import, importall, if, in,%
      macro, module, otherwise, quote, return, switch, true, try, type, typealias,%
      using, while},%
   sensitive=true,%
   alsoother={$},%
   morecomment=[l]\#,%
   morecomment=[n]{\#=}{=\#},%
   morestring=[s]{"}{"},%
   morestring=[m]{'}{'},%
}[keywords,comments,strings]%
\lstset{%
    language         = Julia,
    basicstyle       = \ttfamily,
    keywordstyle     = \bfseries\color{blue},
    stringstyle      = \color{magenta},
    commentstyle     = \color{ForestGreen},
    showstringspaces = false,
}

\title{Modeling and Algorithms for Prof Shit's Project}
\author{Hongda Li, Yining Zhou, Xiaoping Shi}

\begin{document}
\internallinenumbers


\maketitle

\begin{abstract}
    We propose some better algorithm for a problem in detecting structure of probability transition matrices from data. 
\end{abstract}


\section{Introduction}
    We describe an optimization problem introduce by Prof Shi and his student Yining. 
    To start we define the following quantities for the optimization problem. 
    \begin{itemize}
        \item [1.] $n\in \mathbb N$. It denotes the numer of states for a Markov Chain. 
        \item [2.] $p\in \mathbb R^{n\times n}$ denotes the probability transition matrix. It's in small case because it's also the variable for the optimization problem. It supports 2 types of indexing, $p_{ij}$ for $i, j \in \{1, \cdots, n\}$, or $p_j$ with $j\in \{1, \cdots, n^2\}$. More on this later. 
        \item [3.] $\eta_{ij} \ge 0$ for $i, j \in \{1, \cdots, n\}$ is a parameter of the problem. 
        \item [4.] $\hat p$ is the empirically measured probability transition matrix. They are the maximal likelihood estimators for the transition probability in the transition probability matrix. 
        \item [5.] $\lambda$ is the regularization parameter. 
        \item [6.] $\mathbf C^m_n$ is the combinatoric term that counts all possible subset of size $n, n<m$ from a super set of size $m$. 
    \end{itemize}
    \par
\begin{linenumbers}
    When $p$ is referred to as a vector we may say $p \in \mathbb R^{n^2}$, if it's referred to as the matrix, we will use $p \in \mathbb R^{n\times n}$. 
    When indexing $p$ using a tuple, or a single number, it's possible to translate between the two type of indexing scheme using the following bijective map: 
\end{linenumbers}
    \begin{align*}
        & (i, j) \mapsto k:= i \times n + j
        \\
        & k \mapsto (i, j) := (
            \lfloor k/n\rfloor, \text{mod}(k, n) + 1
        ). 
    \end{align*}
    We emphasize, in different programming languages and development environments, the convention of indexing a muti-array using different kind of tuples can be very different. 
    For now we use the aboe indexing, which is a row major index convention (Like Python). 
    \subsection{The Optimization Problem}
    \begin{linenumbers}
        The optimization problem is posed as 
    \end{linenumbers}
        \begin{align}
            &\underset{p \in \mathbb R^{n^2}}{\text{argmin}} 
            \left\lbrace
                \sum_{i = 1}^{n}\sum_{j = 1}^{n}
                    -\eta_{ij}\log(p_{i, j})
                + 
                \lambda \sum_{i=1}^{n^2}\sum_{\substack{j=1\\j\neq i}}^{n^2}
                \frac{1}{2}
                \frac{|p_{i} - p_j|}{|\hat p_{i} - \hat p_j|}
                \right\rbrace
                \\
                &\text{ s.t: }
                \left\lbrace
                \begin{aligned}
                    \sum_{i = 1}^{n} p_{i, j} &= 1 \; && \forall i =1, \cdots, n 
                    \\
                    p_{i, j} &\ge 0\; &&\forall i= 1,\cdots, n, j = 1,\cdots, n 
                \end{aligned}
                \right\rbrace
            \label{eqn:original_formulaion}
        \end{align}
        There are 3 parts to the optimization problem posed above. 
        It has a smooth differentiable function, the sume of $\eta_{ij}\log(p_{i,j})$ but its gradient is not Lipschitz. 
        The ias a non-smooth part with $p_i - p_j$ for all indices $1,\cdots, n^2$ and $i\neq j$. 
        If $\hat p_i - \hat p_j = 0$, then we would ignore the term. 
        Finally, it has a linear constraints on all $n^2$ variables. 
        $p\in \mathbb R^{n^2}$ is a vector with the structure $\Delta_n\oplus\Delta_n\oplus \cdots \oplus \Delta_n$. 
        Each $\Delta_n$ is a probability simplex. 
        It's defined as $\Delta_n = \{x\in \mathbb R^n_+ : \sum_{i =1}^{n}x_n = 1\}$. 
        For simplicity we just denote using notation $\Delta_{n\times n} = \Delta_n\oplus\Delta_n\oplus \cdots \oplus \Delta_n$. 
        
        

\section{Modeling}
    
    The non-smooth part with the absolute value requires some amount of creativity if we were to use common optimization algorithms. 
    \subsection{Representation of the Non-smooth Part}
        \begin{claim}
            The nonsmooth objective can be model as 
            \begin{align*}
                \lambda \sum_{i=1}^{n^2}\sum_{\substack{j> i}}^{n^2}
                \frac{|p_{i} - p_j|}{|\hat p_{i} - \hat p_j|}
                &= 
                \left\Vert
                    Cp
                \right\Vert_1 \quad 
                \text{ where $C$ is $\mathbf C_2^{n^2}$ by $n^2$}. 
            \end{align*}
            The one norm represents the summation of absolute values. 
            The transformation $\mathbf Cp$ is linear transformation and it's a vector of length $\mathbf C_2^{n^2}$. 
            The vector is long and it has a dimension of $(1/2)(1 +n^2)n^2$. 
            Each term inside of the summation is a row of the matrix $C$. 
            Each row of matrix $C$ has exactly 2 non-zero elements in it. 
            Suppose that $i\in \{1,\cdots,\mathbf C^{n^2}_2\}$ denoting the index for a specific row of matrix $C$ denotes the index of a specific row, and $j\in \{1, \cdots, n^2\}$ denotes a specific column of matrix $C$. 
            Mathematically describing the matrix is difficult, but it can be algorithmically defined.
            A sparse matrix format can be described by as a mapping from $(i,j)$, the set of all indices to the element in the vector. 
            The following 
            \hyperref[alg:matrixmk]{algorithm \ref{alg:matrixmk}}
            construct such a mapping. 
            \begin{algorithm}[H]
                \begin{algorithmic}[1]
                \STATE{Let $C$ be a $\mathbf C_2^{n^2}$ by $n^2$ zero matrix. }
                \FOR{$i = 1,\cdots, n^2$}
                    \FOR{$j = 1,\cdots, n^2$}
                        \IF{$|\hat p_i - \hat p_j| == 0$} 
                            \STATE{
                                \textbf{Break}
                            }
                        \ENDIF
                        \STATE{
                            $C[i\times n^2 + j, i] := \lambda/|\hat p_{i} - \hat p_{j}|$
                            } 
                        \STATE{
                            $C[i\times n^2 + j, j]:= -\lambda/|\hat p_{i} - \hat p_{j}|$ 
                        }
                    \ENDFOR
                \ENDFOR
                \end{algorithmic}
                \caption{Matrix Make Algorithm}
                \label{alg:matrixmk}
            \end{algorithm}
            
        \end{claim}
        \begin{remark}
            In practice, we should use sparse matrix data format such as SCR, SCC in programming languages. 
        \end{remark}
    \subsection{Modeling it for Sequential Quadratic Programming}
    \begin{linenumbers}
        To use sequential programming, we model the non-smooth $\Vert Cp\Vert_1$ parts of the objective function as a linear constraints. 
        Define $u\in \mathbb R^{\mathbf C^{n^2}_2}_+$ then the below problem is equivalent to the original formulation:
    \end{linenumbers}
        \begin{align}
            &\underset{
                \substack{
                        p, u
                    }
            }{\text{argmin}} 
            \left\lbrace
                \sum_{i = 1}^{n}\sum_{j = 1}^{n}
                    -\eta_{ij}\log(p_{i, j})
                + 
                \sum_{i = 1}^{C_2^{n^2}} u_k
            \right\rbrace
            \\
            &\text{ s.t: }
            \left\lbrace
            \begin{aligned}
                - u \le Cp \le u
                \\
                p \in \Delta_{n\times n}
                \\ 
                u \in \mathbb R^{\mathbf C^{n^2}_2}_+
            \end{aligned}
            \right\rbrace
        \end{align}
        This is a Non-linear programming problem and it has a convex objective. 
        Common NLP packages in programming languages can solve this efficiently. 
        However it's potentially possible that these solvers are not adapated to huge sparse matrix $C$ that has special structure to it. 
        \begin{remark}
            To formulate into a linear programming with some relaxations, consider that the non-linear objective $-\eta_{ij}\log(p_{i,j})$ can be discretized. 
        \end{remark}
    \subsection{Modeling it for Operator Splitting}
        Operator splitting method aims for objective function of the type $f + g$ where $f, g$ are both convex function and they are proximal friendly. 
        And $\text{ri.dom}(f)\cap \text{ri.dom}(g) \neq \emptyset$. 
        Recall that proximal operator of function $f$ is the resolvent operator on the subgradient $(I + \partial f)^{-1}$. 
        Subgradient is just a generalize type of gradient that can handle continuous function that is not necessarily differentiable. 
        To model the original formulation in such a form, we introduce these quantities: 
        \begin{itemize}
            \item [1.] $f_1(x_1): \mathbb R^{n^2}\mapsto \mathbb R := \sum_{i = 1}^{n^2}-\eta_{i, j}(p_{i, j})$
            \item [2.] $f_2(x_2): \mathbb R^{\mathbf C^{n^2}_2}\mapsto \mathbb R := \Vert x_2\Vert_1$. 
            \item [3.] $f_3(x_3)= \mathbb R^{n^2}\mapsto \mathbb {\bar R}:= \delta_{\Delta_{n\times n}}(x_3)$
        \end{itemize}
        The above 3 functions models the three parts of the objective function. 
        Let $f(x_1, x_2, x_3):= f_1(x_1) + f_2(x_2) + f_3(x_3)$. 
        However, they share different variables $x_1, x_2, x_3$, from different dimension. 
        We want $g$ to represents the constraints that $x_2 = Cp$, and $x_1 = x_3 = p$. 
        Simplifying: $x_2 = C x_1, x_1 = x_3$. 
        This is a linear system of the form
        


\appendix

\bibliographystyle{IEEEtran}
\bibliography{refs.bib}


\end{document}