\documentclass[]{article}
\usepackage{amsmath}
\usepackage{amsfonts} 
\usepackage[english]{babel}
\usepackage{amsthm}
\usepackage{mathtools}
\usepackage{subcaption}
\usepackage{hyperref}
\usepackage{algorithmic}
\usepackage{algorithm}
\usepackage{anyfontsize} % fix font size warning. 
\usepackage{url} 
\urlstyle{same} % fix wacky url links in bib entries. 
% \usepackage{minted}

% Basic Type Settings ----------------------------------------------------------
\usepackage[margin=1in,footskip=0.25in]{geometry}
\linespread{1}  % double spaced or single spaced
\usepackage[fontsize=12pt]{fontsize}
\usepackage{authblk}

\theoremstyle{definition}
\newtheorem{theorem}{Theorem}       % Theorem counter global 
\newtheorem{prop}{Proposition}[section]  % proposition counter is section
\newtheorem{lemma}{Lemma}[subsection]  % lemma counter is subsection
\newtheorem{definition}{Definition}
\newtheorem{remark}{Remark}[subsection]
{
    % \theoremstyle{plain}
    \newtheorem{assumption}{Assumption}
}
\numberwithin{equation}{subsection}

\hypersetup{
    colorlinks=true,
    linkcolor=blue,
    filecolor=magenta,
    urlcolor=cyan,
}
\usepackage[final]{graphicx}
\usepackage{listings}
\usepackage{courier}
\lstset{basicstyle=\footnotesize\ttfamily,breaklines=true}
\newcommand{\indep}{\perp \!\!\! \perp}
\usepackage{wrapfig}
\graphicspath{{.}}
\usepackage{fancyvrb}

%%
%% Julia definition (c) 2014 Jubobs
%%
\usepackage[T1]{fontenc}
\usepackage{beramono}
\usepackage[usenames,dvipsnames]{xcolor}
\lstdefinelanguage{Julia}%
  {morekeywords={abstract,break,case,catch,const,continue,do, else, elseif,%
      end, export, false, for, function, immutable, import, importall, if, in,%
      macro, module, otherwise, quote, return, switch, true, try, type, typealias,%
      using, while},%
   sensitive=true,%
   alsoother={$},%
   morecomment=[l]\#,%
   morecomment=[n]{\#=}{=\#},%
   morestring=[s]{"}{"},%
   morestring=[m]{'}{'},%
}[keywords,comments,strings]%
\lstset{%
    language         = Julia,
    basicstyle       = \ttfamily,
    keywordstyle     = \bfseries\color{blue},
    stringstyle      = \color{magenta},
    commentstyle     = \color{ForestGreen},
    showstringspaces = false,
}

\title{Modeling and Algorithms for Prof Shit's Project}
\author{Hongda Li, Yining Zhou}

\begin{document}
\maketitle

\begin{abstract}
    We propose some better algorithm for a problem in detecting structure of probability transition matrices from data. 
\end{abstract}


\section{Introduction}
    We describe an optimization problem introduce by Prof Shi and his student Yining. 
    To start we define the following quantities for the optimization problem. 
    \begin{enumerate}
        \item [1.] $n\in \mathbb N$. It denotes the numer of states for a Markov Chain. 
        \item [2.] $p\in \mathbb R^{n\times n}$ denotes the probability transition matrix. It's in small case because it's also the variable for the optimization problem. It supports 2 types of indexing, $p_{ij}$ for $i, j \in \{1, \cdots, n\}$, or $p_j$ with $j\in \{1, \cdots, n^2\}$. More on this later. 
        \item [3.] $\eta_{ij} \ge 0$ for $i, j \in \{1, \cdots, n\}$ is a parameter of the problem. 
        \item [4.] $\hat p$ is the empirically measured probability transition matrix. They are the maximal likelihood estimators for the transition probability in the transition probability matrix. 
        \item [5.] $\lambda$ is the regularization parameter. 
    \end{enumerate}
    When $p$ is referred to as a vector we may say $p \in \mathbb R^{n^2}$, if it's referred to as the matrix, we will use $p \in \mathbb R^{n\times n}$. 
    When indexing $p$ using a tuple, or a single number, it's possible to translate between the two type of indexing scheme using the following bijective map: 
    \begin{align*}
        & (i, j) \mapsto k:= i \times n + j
        \\
        & k \mapsto (i, j) := (
            \lfloor k/n\rfloor, \text{mod}(k, n) + 1
        ). 
    \end{align*}
    We emphasize, in different programming languages and development environments, the convention of indexing a muti-array using different kind of tuples can be very different. 
    For now we use the aboe indexing, which is a row major index convention (Like Python). 
    \subsection{The Optimization Problem}

\section{Preliminaries}\label{sec:preliminaries}
    \hyperref[sec:preliminaries]{This} is the preliminary (hyperref without text labeling). 
    
\section{Blah Blah Bleeeh}
    % \begin{algorithm}
    %     \begin{algorithmic}[t]
    %         \STATE{\textbf{Input: $X^{(t)}$}}
    %         \STATE{$Y^{(t)} \sim q (\cdot | X^{(t)})$}
    %         \STATE{
    %             $ 
    %             \rho(x, y) := 
    %             \min\left\lbrace
    %                 \frac{f(y)}{f(x)}\frac{q(x|y)}{q(y|x)}, 1
    %             \right\rbrace
    %             $ 
    %         }
    %         \STATE{
    %             $
    %             X^{(t + 1)} := 
    %             \begin{cases}
    %                 Y^{(t)} & \text{w.p}:  \rho(X^{(t)}, Y^{(t)})
    %                 \\
    %                 X^{(t)} &  \text{otherwise}
    %             \end{cases}$
    %         }
    %     \end{algorithmic}
    %     \caption{Metropolis Chain}
    %     \label{alg:mhc}
    % \end{algorithm}
    % \label{code:brain_expand}
    % \lstinputlisting[language=julia, basicstyle=\ttfamily\scriptsize,numbers=left]{Code/juliacode.jl}
    

    
        
\appendix

\section{Bleeh Bleeh Bleeh I am not Listening}
    This is the Bleeh Bleeh Bleeh I am not Listening section. 

\section{This section is in another .tex file} 
    \input{Sections/section.tex}



\bibliographystyle{IEEETran}
\bibliography{refs.bib}


\end{document}